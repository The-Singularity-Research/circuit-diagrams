
    




    
\documentclass[11pt]{article}

    \usepackage{tikz}
    \usetikzlibrary{quantikz}
    \usepackage[breakable]{tcolorbox}
    \tcbset{nobeforeafter} % prevents tcolorboxes being placing in paragraphs
    \usepackage{float}
    \floatplacement{figure}{H} % forces figures to be placed at the correct location
    
    \usepackage[T1]{fontenc}
    % Nicer default font (+ math font) than Computer Modern for most use cases
    \usepackage{mathpazo}

    % Basic figure setup, for now with no caption control since it's done
    % automatically by Pandoc (which extracts ![](path) syntax from Markdown).
    \usepackage{graphicx}
    % We will generate all images so they have a width \maxwidth. This means
    % that they will get their normal width if they fit onto the page, but
    % are scaled down if they would overflow the margins.
    \makeatletter
    \def\maxwidth{\ifdim\Gin@nat@width>\linewidth\linewidth
    \else\Gin@nat@width\fi}
    \makeatother
    \let\Oldincludegraphics\includegraphics
    % Set max figure width to be 80% of text width, for now hardcoded.
    \renewcommand{\includegraphics}[1]{\Oldincludegraphics[width=.8\maxwidth]{#1}}
    % Ensure that by default, figures have no caption (until we provide a
    % proper Figure object with a Caption API and a way to capture that
    % in the conversion process - todo).
    \usepackage{caption}
    \DeclareCaptionLabelFormat{nolabel}{}
    \captionsetup{labelformat=nolabel}

    \usepackage{adjustbox} % Used to constrain images to a maximum size 
    \usepackage{xcolor} % Allow colors to be defined
    \usepackage{enumerate} % Needed for markdown enumerations to work
    \usepackage{geometry} % Used to adjust the document margins
    \usepackage{amsmath} % Equations
    \usepackage{amssymb} % Equations
    \usepackage{textcomp} % defines textquotesingle
    % Hack from http://tex.stackexchange.com/a/47451/13684:
    \AtBeginDocument{%
        \def\PYZsq{\textquotesingle}% Upright quotes in Pygmentized code
    }
    \usepackage{upquote} % Upright quotes for verbatim code
    \usepackage{eurosym} % defines \euro
    \usepackage[mathletters]{ucs} % Extended unicode (utf-8) support
    \usepackage[utf8x]{inputenc} % Allow utf-8 characters in the tex document
    \usepackage{fancyvrb} % verbatim replacement that allows latex
    \usepackage{grffile} % extends the file name processing of package graphics 
                         % to support a larger range 
    % The hyperref package gives us a pdf with properly built
    % internal navigation ('pdf bookmarks' for the table of contents,
    % internal cross-reference links, web links for URLs, etc.)
    \usepackage{hyperref}
    \usepackage{longtable} % longtable support required by pandoc >1.10
    \usepackage{booktabs}  % table support for pandoc > 1.12.2
    \usepackage[inline]{enumitem} % IRkernel/repr support (it uses the enumerate* environment)
    \usepackage[normalem]{ulem} % ulem is needed to support strikethroughs (\sout)
                                % normalem makes italics be italics, not underlines
    \usepackage{mathrsfs}
    

    
    % Colors for the hyperref package
    \definecolor{urlcolor}{rgb}{0,.145,.698}
    \definecolor{linkcolor}{rgb}{.71,0.21,0.01}
    \definecolor{citecolor}{rgb}{.12,.54,.11}

    % ANSI colors
    \definecolor{ansi-black}{HTML}{3E424D}
    \definecolor{ansi-black-intense}{HTML}{282C36}
    \definecolor{ansi-red}{HTML}{E75C58}
    \definecolor{ansi-red-intense}{HTML}{B22B31}
    \definecolor{ansi-green}{HTML}{00A250}
    \definecolor{ansi-green-intense}{HTML}{007427}
    \definecolor{ansi-yellow}{HTML}{DDB62B}
    \definecolor{ansi-yellow-intense}{HTML}{B27D12}
    \definecolor{ansi-blue}{HTML}{208FFB}
    \definecolor{ansi-blue-intense}{HTML}{0065CA}
    \definecolor{ansi-magenta}{HTML}{D160C4}
    \definecolor{ansi-magenta-intense}{HTML}{A03196}
    \definecolor{ansi-cyan}{HTML}{60C6C8}
    \definecolor{ansi-cyan-intense}{HTML}{258F8F}
    \definecolor{ansi-white}{HTML}{C5C1B4}
    \definecolor{ansi-white-intense}{HTML}{A1A6B2}
    \definecolor{ansi-default-inverse-fg}{HTML}{FFFFFF}
    \definecolor{ansi-default-inverse-bg}{HTML}{000000}

    % commands and environments needed by pandoc snippets
    % extracted from the output of `pandoc -s`
    \providecommand{\tightlist}{%
      \setlength{\itemsep}{0pt}\setlength{\parskip}{0pt}}
    \DefineVerbatimEnvironment{Highlighting}{Verbatim}{commandchars=\\\{\}}
    % Add ',fontsize=\small' for more characters per line
    \newenvironment{Shaded}{}{}
    \newcommand{\KeywordTok}[1]{\textcolor[rgb]{0.00,0.44,0.13}{\textbf{{#1}}}}
    \newcommand{\DataTypeTok}[1]{\textcolor[rgb]{0.56,0.13,0.00}{{#1}}}
    \newcommand{\DecValTok}[1]{\textcolor[rgb]{0.25,0.63,0.44}{{#1}}}
    \newcommand{\BaseNTok}[1]{\textcolor[rgb]{0.25,0.63,0.44}{{#1}}}
    \newcommand{\FloatTok}[1]{\textcolor[rgb]{0.25,0.63,0.44}{{#1}}}
    \newcommand{\CharTok}[1]{\textcolor[rgb]{0.25,0.44,0.63}{{#1}}}
    \newcommand{\StringTok}[1]{\textcolor[rgb]{0.25,0.44,0.63}{{#1}}}
    \newcommand{\CommentTok}[1]{\textcolor[rgb]{0.38,0.63,0.69}{\textit{{#1}}}}
    \newcommand{\OtherTok}[1]{\textcolor[rgb]{0.00,0.44,0.13}{{#1}}}
    \newcommand{\AlertTok}[1]{\textcolor[rgb]{1.00,0.00,0.00}{\textbf{{#1}}}}
    \newcommand{\FunctionTok}[1]{\textcolor[rgb]{0.02,0.16,0.49}{{#1}}}
    \newcommand{\RegionMarkerTok}[1]{{#1}}
    \newcommand{\ErrorTok}[1]{\textcolor[rgb]{1.00,0.00,0.00}{\textbf{{#1}}}}
    \newcommand{\NormalTok}[1]{{#1}}
    
    % Additional commands for more recent versions of Pandoc
    \newcommand{\ConstantTok}[1]{\textcolor[rgb]{0.53,0.00,0.00}{{#1}}}
    \newcommand{\SpecialCharTok}[1]{\textcolor[rgb]{0.25,0.44,0.63}{{#1}}}
    \newcommand{\VerbatimStringTok}[1]{\textcolor[rgb]{0.25,0.44,0.63}{{#1}}}
    \newcommand{\SpecialStringTok}[1]{\textcolor[rgb]{0.73,0.40,0.53}{{#1}}}
    \newcommand{\ImportTok}[1]{{#1}}
    \newcommand{\DocumentationTok}[1]{\textcolor[rgb]{0.73,0.13,0.13}{\textit{{#1}}}}
    \newcommand{\AnnotationTok}[1]{\textcolor[rgb]{0.38,0.63,0.69}{\textbf{\textit{{#1}}}}}
    \newcommand{\CommentVarTok}[1]{\textcolor[rgb]{0.38,0.63,0.69}{\textbf{\textit{{#1}}}}}
    \newcommand{\VariableTok}[1]{\textcolor[rgb]{0.10,0.09,0.49}{{#1}}}
    \newcommand{\ControlFlowTok}[1]{\textcolor[rgb]{0.00,0.44,0.13}{\textbf{{#1}}}}
    \newcommand{\OperatorTok}[1]{\textcolor[rgb]{0.40,0.40,0.40}{{#1}}}
    \newcommand{\BuiltInTok}[1]{{#1}}
    \newcommand{\ExtensionTok}[1]{{#1}}
    \newcommand{\PreprocessorTok}[1]{\textcolor[rgb]{0.74,0.48,0.00}{{#1}}}
    \newcommand{\AttributeTok}[1]{\textcolor[rgb]{0.49,0.56,0.16}{{#1}}}
    \newcommand{\InformationTok}[1]{\textcolor[rgb]{0.38,0.63,0.69}{\textbf{\textit{{#1}}}}}
    \newcommand{\WarningTok}[1]{\textcolor[rgb]{0.38,0.63,0.69}{\textbf{\textit{{#1}}}}}
    
    
    % Define a nice break command that doesn't care if a line doesn't already
    % exist.
    \def\br{\hspace*{\fill} \\* }
    % Math Jax compatibility definitions
    \def\gt{>}
    \def\lt{<}
    \let\Oldtex\TeX
    \let\Oldlatex\LaTeX
    \renewcommand{\TeX}{\textrm{\Oldtex}}
    \renewcommand{\LaTeX}{\textrm{\Oldlatex}}
    % Document parameters
    % Document title
    \title{circuit\_diagrams}
    
    
    
    
    
% Pygments definitions
\makeatletter
\def\PY@reset{\let\PY@it=\relax \let\PY@bf=\relax%
    \let\PY@ul=\relax \let\PY@tc=\relax%
    \let\PY@bc=\relax \let\PY@ff=\relax}
\def\PY@tok#1{\csname PY@tok@#1\endcsname}
\def\PY@toks#1+{\ifx\relax#1\empty\else%
    \PY@tok{#1}\expandafter\PY@toks\fi}
\def\PY@do#1{\PY@bc{\PY@tc{\PY@ul{%
    \PY@it{\PY@bf{\PY@ff{#1}}}}}}}
\def\PY#1#2{\PY@reset\PY@toks#1+\relax+\PY@do{#2}}

\expandafter\def\csname PY@tok@w\endcsname{\def\PY@tc##1{\textcolor[rgb]{0.73,0.73,0.73}{##1}}}
\expandafter\def\csname PY@tok@c\endcsname{\let\PY@it=\textit\def\PY@tc##1{\textcolor[rgb]{0.25,0.50,0.50}{##1}}}
\expandafter\def\csname PY@tok@cp\endcsname{\def\PY@tc##1{\textcolor[rgb]{0.74,0.48,0.00}{##1}}}
\expandafter\def\csname PY@tok@k\endcsname{\let\PY@bf=\textbf\def\PY@tc##1{\textcolor[rgb]{0.00,0.50,0.00}{##1}}}
\expandafter\def\csname PY@tok@kp\endcsname{\def\PY@tc##1{\textcolor[rgb]{0.00,0.50,0.00}{##1}}}
\expandafter\def\csname PY@tok@kt\endcsname{\def\PY@tc##1{\textcolor[rgb]{0.69,0.00,0.25}{##1}}}
\expandafter\def\csname PY@tok@o\endcsname{\def\PY@tc##1{\textcolor[rgb]{0.40,0.40,0.40}{##1}}}
\expandafter\def\csname PY@tok@ow\endcsname{\let\PY@bf=\textbf\def\PY@tc##1{\textcolor[rgb]{0.67,0.13,1.00}{##1}}}
\expandafter\def\csname PY@tok@nb\endcsname{\def\PY@tc##1{\textcolor[rgb]{0.00,0.50,0.00}{##1}}}
\expandafter\def\csname PY@tok@nf\endcsname{\def\PY@tc##1{\textcolor[rgb]{0.00,0.00,1.00}{##1}}}
\expandafter\def\csname PY@tok@nc\endcsname{\let\PY@bf=\textbf\def\PY@tc##1{\textcolor[rgb]{0.00,0.00,1.00}{##1}}}
\expandafter\def\csname PY@tok@nn\endcsname{\let\PY@bf=\textbf\def\PY@tc##1{\textcolor[rgb]{0.00,0.00,1.00}{##1}}}
\expandafter\def\csname PY@tok@ne\endcsname{\let\PY@bf=\textbf\def\PY@tc##1{\textcolor[rgb]{0.82,0.25,0.23}{##1}}}
\expandafter\def\csname PY@tok@nv\endcsname{\def\PY@tc##1{\textcolor[rgb]{0.10,0.09,0.49}{##1}}}
\expandafter\def\csname PY@tok@no\endcsname{\def\PY@tc##1{\textcolor[rgb]{0.53,0.00,0.00}{##1}}}
\expandafter\def\csname PY@tok@nl\endcsname{\def\PY@tc##1{\textcolor[rgb]{0.63,0.63,0.00}{##1}}}
\expandafter\def\csname PY@tok@ni\endcsname{\let\PY@bf=\textbf\def\PY@tc##1{\textcolor[rgb]{0.60,0.60,0.60}{##1}}}
\expandafter\def\csname PY@tok@na\endcsname{\def\PY@tc##1{\textcolor[rgb]{0.49,0.56,0.16}{##1}}}
\expandafter\def\csname PY@tok@nt\endcsname{\let\PY@bf=\textbf\def\PY@tc##1{\textcolor[rgb]{0.00,0.50,0.00}{##1}}}
\expandafter\def\csname PY@tok@nd\endcsname{\def\PY@tc##1{\textcolor[rgb]{0.67,0.13,1.00}{##1}}}
\expandafter\def\csname PY@tok@s\endcsname{\def\PY@tc##1{\textcolor[rgb]{0.73,0.13,0.13}{##1}}}
\expandafter\def\csname PY@tok@sd\endcsname{\let\PY@it=\textit\def\PY@tc##1{\textcolor[rgb]{0.73,0.13,0.13}{##1}}}
\expandafter\def\csname PY@tok@si\endcsname{\let\PY@bf=\textbf\def\PY@tc##1{\textcolor[rgb]{0.73,0.40,0.53}{##1}}}
\expandafter\def\csname PY@tok@se\endcsname{\let\PY@bf=\textbf\def\PY@tc##1{\textcolor[rgb]{0.73,0.40,0.13}{##1}}}
\expandafter\def\csname PY@tok@sr\endcsname{\def\PY@tc##1{\textcolor[rgb]{0.73,0.40,0.53}{##1}}}
\expandafter\def\csname PY@tok@ss\endcsname{\def\PY@tc##1{\textcolor[rgb]{0.10,0.09,0.49}{##1}}}
\expandafter\def\csname PY@tok@sx\endcsname{\def\PY@tc##1{\textcolor[rgb]{0.00,0.50,0.00}{##1}}}
\expandafter\def\csname PY@tok@m\endcsname{\def\PY@tc##1{\textcolor[rgb]{0.40,0.40,0.40}{##1}}}
\expandafter\def\csname PY@tok@gh\endcsname{\let\PY@bf=\textbf\def\PY@tc##1{\textcolor[rgb]{0.00,0.00,0.50}{##1}}}
\expandafter\def\csname PY@tok@gu\endcsname{\let\PY@bf=\textbf\def\PY@tc##1{\textcolor[rgb]{0.50,0.00,0.50}{##1}}}
\expandafter\def\csname PY@tok@gd\endcsname{\def\PY@tc##1{\textcolor[rgb]{0.63,0.00,0.00}{##1}}}
\expandafter\def\csname PY@tok@gi\endcsname{\def\PY@tc##1{\textcolor[rgb]{0.00,0.63,0.00}{##1}}}
\expandafter\def\csname PY@tok@gr\endcsname{\def\PY@tc##1{\textcolor[rgb]{1.00,0.00,0.00}{##1}}}
\expandafter\def\csname PY@tok@ge\endcsname{\let\PY@it=\textit}
\expandafter\def\csname PY@tok@gs\endcsname{\let\PY@bf=\textbf}
\expandafter\def\csname PY@tok@gp\endcsname{\let\PY@bf=\textbf\def\PY@tc##1{\textcolor[rgb]{0.00,0.00,0.50}{##1}}}
\expandafter\def\csname PY@tok@go\endcsname{\def\PY@tc##1{\textcolor[rgb]{0.53,0.53,0.53}{##1}}}
\expandafter\def\csname PY@tok@gt\endcsname{\def\PY@tc##1{\textcolor[rgb]{0.00,0.27,0.87}{##1}}}
\expandafter\def\csname PY@tok@err\endcsname{\def\PY@bc##1{\setlength{\fboxsep}{0pt}\fcolorbox[rgb]{1.00,0.00,0.00}{1,1,1}{\strut ##1}}}
\expandafter\def\csname PY@tok@kc\endcsname{\let\PY@bf=\textbf\def\PY@tc##1{\textcolor[rgb]{0.00,0.50,0.00}{##1}}}
\expandafter\def\csname PY@tok@kd\endcsname{\let\PY@bf=\textbf\def\PY@tc##1{\textcolor[rgb]{0.00,0.50,0.00}{##1}}}
\expandafter\def\csname PY@tok@kn\endcsname{\let\PY@bf=\textbf\def\PY@tc##1{\textcolor[rgb]{0.00,0.50,0.00}{##1}}}
\expandafter\def\csname PY@tok@kr\endcsname{\let\PY@bf=\textbf\def\PY@tc##1{\textcolor[rgb]{0.00,0.50,0.00}{##1}}}
\expandafter\def\csname PY@tok@bp\endcsname{\def\PY@tc##1{\textcolor[rgb]{0.00,0.50,0.00}{##1}}}
\expandafter\def\csname PY@tok@fm\endcsname{\def\PY@tc##1{\textcolor[rgb]{0.00,0.00,1.00}{##1}}}
\expandafter\def\csname PY@tok@vc\endcsname{\def\PY@tc##1{\textcolor[rgb]{0.10,0.09,0.49}{##1}}}
\expandafter\def\csname PY@tok@vg\endcsname{\def\PY@tc##1{\textcolor[rgb]{0.10,0.09,0.49}{##1}}}
\expandafter\def\csname PY@tok@vi\endcsname{\def\PY@tc##1{\textcolor[rgb]{0.10,0.09,0.49}{##1}}}
\expandafter\def\csname PY@tok@vm\endcsname{\def\PY@tc##1{\textcolor[rgb]{0.10,0.09,0.49}{##1}}}
\expandafter\def\csname PY@tok@sa\endcsname{\def\PY@tc##1{\textcolor[rgb]{0.73,0.13,0.13}{##1}}}
\expandafter\def\csname PY@tok@sb\endcsname{\def\PY@tc##1{\textcolor[rgb]{0.73,0.13,0.13}{##1}}}
\expandafter\def\csname PY@tok@sc\endcsname{\def\PY@tc##1{\textcolor[rgb]{0.73,0.13,0.13}{##1}}}
\expandafter\def\csname PY@tok@dl\endcsname{\def\PY@tc##1{\textcolor[rgb]{0.73,0.13,0.13}{##1}}}
\expandafter\def\csname PY@tok@s2\endcsname{\def\PY@tc##1{\textcolor[rgb]{0.73,0.13,0.13}{##1}}}
\expandafter\def\csname PY@tok@sh\endcsname{\def\PY@tc##1{\textcolor[rgb]{0.73,0.13,0.13}{##1}}}
\expandafter\def\csname PY@tok@s1\endcsname{\def\PY@tc##1{\textcolor[rgb]{0.73,0.13,0.13}{##1}}}
\expandafter\def\csname PY@tok@mb\endcsname{\def\PY@tc##1{\textcolor[rgb]{0.40,0.40,0.40}{##1}}}
\expandafter\def\csname PY@tok@mf\endcsname{\def\PY@tc##1{\textcolor[rgb]{0.40,0.40,0.40}{##1}}}
\expandafter\def\csname PY@tok@mh\endcsname{\def\PY@tc##1{\textcolor[rgb]{0.40,0.40,0.40}{##1}}}
\expandafter\def\csname PY@tok@mi\endcsname{\def\PY@tc##1{\textcolor[rgb]{0.40,0.40,0.40}{##1}}}
\expandafter\def\csname PY@tok@il\endcsname{\def\PY@tc##1{\textcolor[rgb]{0.40,0.40,0.40}{##1}}}
\expandafter\def\csname PY@tok@mo\endcsname{\def\PY@tc##1{\textcolor[rgb]{0.40,0.40,0.40}{##1}}}
\expandafter\def\csname PY@tok@ch\endcsname{\let\PY@it=\textit\def\PY@tc##1{\textcolor[rgb]{0.25,0.50,0.50}{##1}}}
\expandafter\def\csname PY@tok@cm\endcsname{\let\PY@it=\textit\def\PY@tc##1{\textcolor[rgb]{0.25,0.50,0.50}{##1}}}
\expandafter\def\csname PY@tok@cpf\endcsname{\let\PY@it=\textit\def\PY@tc##1{\textcolor[rgb]{0.25,0.50,0.50}{##1}}}
\expandafter\def\csname PY@tok@c1\endcsname{\let\PY@it=\textit\def\PY@tc##1{\textcolor[rgb]{0.25,0.50,0.50}{##1}}}
\expandafter\def\csname PY@tok@cs\endcsname{\let\PY@it=\textit\def\PY@tc##1{\textcolor[rgb]{0.25,0.50,0.50}{##1}}}

\def\PYZbs{\char`\\}
\def\PYZus{\char`\_}
\def\PYZob{\char`\{}
\def\PYZcb{\char`\}}
\def\PYZca{\char`\^}
\def\PYZam{\char`\&}
\def\PYZlt{\char`\<}
\def\PYZgt{\char`\>}
\def\PYZsh{\char`\#}
\def\PYZpc{\char`\%}
\def\PYZdl{\char`\$}
\def\PYZhy{\char`\-}
\def\PYZsq{\char`\'}
\def\PYZdq{\char`\"}
\def\PYZti{\char`\~}
% for compatibility with earlier versions
\def\PYZat{@}
\def\PYZlb{[}
\def\PYZrb{]}
\makeatother


    % For linebreaks inside Verbatim environment from package fancyvrb. 
    \makeatletter
        \newbox\Wrappedcontinuationbox 
        \newbox\Wrappedvisiblespacebox 
        \newcommand*\Wrappedvisiblespace {\textcolor{red}{\textvisiblespace}} 
        \newcommand*\Wrappedcontinuationsymbol {\textcolor{red}{\llap{\tiny$\m@th\hookrightarrow$}}} 
        \newcommand*\Wrappedcontinuationindent {3ex } 
        \newcommand*\Wrappedafterbreak {\kern\Wrappedcontinuationindent\copy\Wrappedcontinuationbox} 
        % Take advantage of the already applied Pygments mark-up to insert 
        % potential linebreaks for TeX processing. 
        %        {, <, #, %, $, ' and ": go to next line. 
        %        _, }, ^, &, >, - and ~: stay at end of broken line. 
        % Use of \textquotesingle for straight quote. 
        \newcommand*\Wrappedbreaksatspecials {% 
            \def\PYGZus{\discretionary{\char`\_}{\Wrappedafterbreak}{\char`\_}}% 
            \def\PYGZob{\discretionary{}{\Wrappedafterbreak\char`\{}{\char`\{}}% 
            \def\PYGZcb{\discretionary{\char`\}}{\Wrappedafterbreak}{\char`\}}}% 
            \def\PYGZca{\discretionary{\char`\^}{\Wrappedafterbreak}{\char`\^}}% 
            \def\PYGZam{\discretionary{\char`\&}{\Wrappedafterbreak}{\char`\&}}% 
            \def\PYGZlt{\discretionary{}{\Wrappedafterbreak\char`\<}{\char`\<}}% 
            \def\PYGZgt{\discretionary{\char`\>}{\Wrappedafterbreak}{\char`\>}}% 
            \def\PYGZsh{\discretionary{}{\Wrappedafterbreak\char`\#}{\char`\#}}% 
            \def\PYGZpc{\discretionary{}{\Wrappedafterbreak\char`\%}{\char`\%}}% 
            \def\PYGZdl{\discretionary{}{\Wrappedafterbreak\char`\$}{\char`\$}}% 
            \def\PYGZhy{\discretionary{\char`\-}{\Wrappedafterbreak}{\char`\-}}% 
            \def\PYGZsq{\discretionary{}{\Wrappedafterbreak\textquotesingle}{\textquotesingle}}% 
            \def\PYGZdq{\discretionary{}{\Wrappedafterbreak\char`\"}{\char`\"}}% 
            \def\PYGZti{\discretionary{\char`\~}{\Wrappedafterbreak}{\char`\~}}% 
        } 
        % Some characters . , ; ? ! / are not pygmentized. 
        % This macro makes them "active" and they will insert potential linebreaks 
        \newcommand*\Wrappedbreaksatpunct {% 
            \lccode`\~`\.\lowercase{\def~}{\discretionary{\hbox{\char`\.}}{\Wrappedafterbreak}{\hbox{\char`\.}}}% 
            \lccode`\~`\,\lowercase{\def~}{\discretionary{\hbox{\char`\,}}{\Wrappedafterbreak}{\hbox{\char`\,}}}% 
            \lccode`\~`\;\lowercase{\def~}{\discretionary{\hbox{\char`\;}}{\Wrappedafterbreak}{\hbox{\char`\;}}}% 
            \lccode`\~`\:\lowercase{\def~}{\discretionary{\hbox{\char`\:}}{\Wrappedafterbreak}{\hbox{\char`\:}}}% 
            \lccode`\~`\?\lowercase{\def~}{\discretionary{\hbox{\char`\?}}{\Wrappedafterbreak}{\hbox{\char`\?}}}% 
            \lccode`\~`\!\lowercase{\def~}{\discretionary{\hbox{\char`\!}}{\Wrappedafterbreak}{\hbox{\char`\!}}}% 
            \lccode`\~`\/\lowercase{\def~}{\discretionary{\hbox{\char`\/}}{\Wrappedafterbreak}{\hbox{\char`\/}}}% 
            \catcode`\.\active
            \catcode`\,\active 
            \catcode`\;\active
            \catcode`\:\active
            \catcode`\?\active
            \catcode`\!\active
            \catcode`\/\active 
            \lccode`\~`\~ 	
        }
    \makeatother

    \let\OriginalVerbatim=\Verbatim
    \makeatletter
    \renewcommand{\Verbatim}[1][1]{%
        %\parskip\z@skip
        \sbox\Wrappedcontinuationbox {\Wrappedcontinuationsymbol}%
        \sbox\Wrappedvisiblespacebox {\FV@SetupFont\Wrappedvisiblespace}%
        \def\FancyVerbFormatLine ##1{\hsize\linewidth
            \vtop{\raggedright\hyphenpenalty\z@\exhyphenpenalty\z@
                \doublehyphendemerits\z@\finalhyphendemerits\z@
                \strut ##1\strut}%
        }%
        % If the linebreak is at a space, the latter will be displayed as visible
        % space at end of first line, and a continuation symbol starts next line.
        % Stretch/shrink are however usually zero for typewriter font.
        \def\FV@Space {%
            \nobreak\hskip\z@ plus\fontdimen3\font minus\fontdimen4\font
            \discretionary{\copy\Wrappedvisiblespacebox}{\Wrappedafterbreak}
            {\kern\fontdimen2\font}%
        }%
        
        % Allow breaks at special characters using \PYG... macros.
        \Wrappedbreaksatspecials
        % Breaks at punctuation characters . , ; ? ! and / need catcode=\active 	
        \OriginalVerbatim[#1,codes*=\Wrappedbreaksatpunct]%
    }
    \makeatother

    % Exact colors from NB
    \definecolor{incolor}{HTML}{303F9F}
    \definecolor{outcolor}{HTML}{D84315}
    \definecolor{cellborder}{HTML}{CFCFCF}
    \definecolor{cellbackground}{HTML}{F7F7F7}
    
    % prompt
    \newcommand{\prompt}[4]{
        \llap{{\color{#2}[#3]: #4}}\vspace{-1.25em}
    }
    

    
    % Prevent overflowing lines due to hard-to-break entities
    \sloppy 
    % Setup hyperref package
    \hypersetup{
      breaklinks=true,  % so long urls are correctly broken across lines
      colorlinks=true,
      urlcolor=urlcolor,
      linkcolor=linkcolor,
      citecolor=citecolor,
      }
    % Slightly bigger margins than the latex defaults
    
    \geometry{verbose,tmargin=1in,bmargin=1in,lmargin=1in,rmargin=1in}
    
    

    \begin{document}
    
    
    \maketitle
    
    

    
    \hypertarget{quantum-circuits}{%
\section{Quantum Circuits}\label{quantum-circuits}}

    \hypertarget{dependencies-and-imports}{%
\subsection{Dependencies and Imports}\label{dependencies-and-imports}}

    \begin{tcolorbox}[breakable, size=fbox, boxrule=1pt, pad at break*=1mm,colback=cellbackground, colframe=cellborder]
\prompt{In}{incolor}{1}{\hspace{4pt}}
\begin{Verbatim}[commandchars=\\\{\}]
\PY{c+c1}{\PYZsh{} Read packages into Python library:}
\PY{k+kn}{import} \PY{n+nn}{pennylane} \PY{k}{as} \PY{n+nn}{qml}
\PY{k+kn}{from} \PY{n+nn}{pennylane} \PY{k}{import} \PY{n}{numpy} \PY{k}{as} \PY{n}{np}
\end{Verbatim}
\end{tcolorbox}

    \hypertarget{introduction}{%
\subsection{Introduction}\label{introduction}}

Now that we have all of the basics of linear algebra and tensor products
out of the way, let's have a look at multi-qubit gates in more depth and
in the context of quantum circuits. Quantum circuits are diagrams that
show how quantum logic gates act on qubits. They are a special case of a
more general structure called \textbf{tensor networks}. We will talk
about tensor networks, quantum complexity via entanglement entropy, and
some of the historical motivations from various areas of physics in the
appendix. For now, let's focus on some of the basics and how to
interpret quantum circuit diagrams.

    \hypertarget{circuit-diagrams-of-common-gates}{%
\subsection{Circuit Diagrams of Common
Gates}\label{circuit-diagrams-of-common-gates}}

    In this section we will use the operators listed below in circuit
diagrams in order to learn to convert quantum circuit diagrams into
linear algebra operations.

    \hypertarget{pauli-gates}{%
\subsubsection{Pauli Gates}\label{pauli-gates}}

\begin{tcolorbox}
\begin{center}
\begin{align}
X = \begin{pmatrix}0&1 \\ 1&0\end{pmatrix}, \quad
Y=\begin{pmatrix}0&-i \\ i&0\end{pmatrix}, \quad 
Z=\begin{pmatrix}1&0 \\ 0&-1 \end{pmatrix}, \quad
I=\begin{pmatrix}1&0 \\ 0&1 \end{pmatrix} 
\end{align}


\begin{quantikz}
 \qw & \gate{X} & \qw 
\end{quantikz}

\begin{quantikz}
 \qw & \gate{Y} & \qw 
\end{quantikz}

\begin{quantikz}
 \qw & \gate{Z} & \qw 
\end{quantikz}

\begin{quantikz}
 \qw & \gate{I} & \qw 
\end{quantikz}
\end{center}
\end{tcolorbox}

    \hypertarget{the-hadamard-gate}{%
\subsubsection{The Hadamard Gate}\label{the-hadamard-gate}}

\begin{tcolorbox}
\begin{center}
\begin{align}
H = \frac{1}{\sqrt{2}}
\begin{pmatrix}1 & 1 \\ 1 & -1 \end{pmatrix}
\end{align}


\begin{quantikz}
 \qw & \gate{H} & \qw 
\end{quantikz}
\end{center}
\end{tcolorbox}

    \hypertarget{the-cnot-gate}{%
\subsubsection{The ``CNOT'' Gate}\label{the-cnot-gate}}

\begin{tcolorbox}
\begin{center}
\begin{align}
\mathbf{CX} = 
\begin{pmatrix} I & 0 \\ 0 & X \end{pmatrix} = \begin{pmatrix} 1 & 0 & 0 & 0\\ 0 & 1 & 0 & 0 \\ 0 & 0 & 0 & 1\\ 0 & 0 & 1 & 0 \end{pmatrix}
\end{align}

\begin{quantikz}
 \qw & \ctrl{1} &  \qw \\
 \qw & \targ{}  &  \qw
\end{quantikz}
\end{center}
\end{tcolorbox}

\bigskip

    As an example, let's try converting the following circuit diagram into
linear algebra:

\begin{center}
\begin{quantikz}
\lstick{\ket{0}} \qw & \gate{H} & \gate{X} & \qw \rstick{$\ket{\psi}$} 
\end{quantikz}
\end{center}

\begin{align}
  X \cdot H |0\rangle &= X \cdot H \cdot \begin{pmatrix}1 \\ 0\end{pmatrix} \\
  &=  \frac{1}{\sqrt{2}}\begin{pmatrix} 0 & 1 \\ 1 & 0\end{pmatrix} \begin{pmatrix} 1 & 1 \\ 1 & -1\end{pmatrix} \begin{pmatrix}1 \\ 0\end{pmatrix} \\
  &= \frac{1}{\sqrt{2}}\begin{pmatrix} 0 & 1 \\ 1 & 0\end{pmatrix} \begin{pmatrix}1\\1 \end{pmatrix} \\
  &= \frac{1}{\sqrt{2}}\begin{pmatrix}1\\1 \end{pmatrix}\\
  &= \begin{pmatrix}1/\sqrt{2}\\1/\sqrt{2} \end{pmatrix}\\
  &= \frac{1}{\sqrt{2}}(|0\rangle + |1\rangle)\\
  &= |r\rangle \quad (\text{spin-right in the X-basis})\\
  &= |\psi \rangle
\end{align}

    In code, we can construct a PennyLane \texttt{qnode()} that performs
these gate operations and then samples the circuit

    \begin{tcolorbox}[breakable, size=fbox, boxrule=1pt, pad at break*=1mm,colback=cellbackground, colframe=cellborder]
\prompt{In}{incolor}{2}{\hspace{4pt}}
\begin{Verbatim}[commandchars=\\\{\}]
\PY{c+c1}{\PYZsh{} Create a device to run the code}
\PY{n}{dev} \PY{o}{=} \PY{n}{qml}\PY{o}{.}\PY{n}{device}\PY{p}{(}\PY{l+s+s2}{\PYZdq{}}\PY{l+s+s2}{default.qubit}\PY{l+s+s2}{\PYZdq{}}\PY{p}{,} \PY{n}{wires}\PY{o}{=}\PY{l+m+mi}{1}\PY{p}{,} \PY{n}{shots}\PY{o}{=}\PY{l+m+mi}{1}\PY{p}{)}

\PY{c+c1}{\PYZsh{}Create the qnode}
\PY{n+nd}{@qml}\PY{o}{.}\PY{n}{qnode}\PY{p}{(}\PY{n}{dev}\PY{p}{)}
\PY{k}{def} \PY{n+nf}{circuit}\PY{p}{(}\PY{p}{)}\PY{p}{:}
    \PY{n}{qml}\PY{o}{.}\PY{n}{Hadamard}\PY{p}{(}\PY{n}{wires}\PY{o}{=}\PY{p}{[}\PY{l+m+mi}{0}\PY{p}{]}\PY{p}{)}
    \PY{n}{qml}\PY{o}{.}\PY{n}{PauliX}\PY{p}{(}\PY{n}{wires}\PY{o}{=}\PY{p}{[}\PY{l+m+mi}{0}\PY{p}{]}\PY{p}{)}
    \PY{k}{return} \PY{n}{qml}\PY{o}{.}\PY{n}{sample}\PY{p}{(}\PY{n}{qml}\PY{o}{.}\PY{n}{PauliX}\PY{p}{(}\PY{l+m+mi}{0}\PY{p}{)}\PY{p}{)}

\PY{n+nb}{print}\PY{p}{(}\PY{n}{circuit}\PY{p}{(}\PY{p}{)}\PY{p}{)}
\end{Verbatim}
\end{tcolorbox}

    \begin{Verbatim}[commandchars=\\\{\}]
1
\end{Verbatim}

    Notice, we compute the sample in the \(X\)-basis by having the
\texttt{qnode()} return \texttt{qml.sample(qml.PauliX(0))}. In the
previous chapter, we always sampled using the \(Z\)-basis by having the
circuit return \texttt{qml.sample(qml.PauliZ(0))} We will discuss this
in more depth when we talk about \textbf{measurements} and
\textbf{expectation values}. Since the result of the circuit gave the
state \begin{align}
|\psi \rangle &= |r \rangle \\
&= \frac{1}{\sqrt{2}}(|0\rangle + |1\rangle)
\end{align}

the sampling returned a ``\(+1\)'' value. If we instead run the same
curcuit, but prepare the intitial state to be \(|1\rangle\) we will get
a different result:

    \begin{tcolorbox}[breakable, size=fbox, boxrule=1pt, pad at break*=1mm,colback=cellbackground, colframe=cellborder]
\prompt{In}{incolor}{3}{\hspace{4pt}}
\begin{Verbatim}[commandchars=\\\{\}]
\PY{c+c1}{\PYZsh{}Define an array corresponding to the initial state |1\PYZgt{}}
\PY{n}{u} \PY{o}{=} \PY{n}{np}\PY{o}{.}\PY{n}{array}\PY{p}{(}\PY{p}{[}\PY{l+m+mi}{1}\PY{p}{]}\PY{p}{)}

\PY{c+c1}{\PYZsh{}Create the qnode}
\PY{n+nd}{@qml}\PY{o}{.}\PY{n}{qnode}\PY{p}{(}\PY{n}{dev}\PY{p}{)}
\PY{k}{def} \PY{n+nf}{circuit}\PY{p}{(}\PY{p}{)}\PY{p}{:}
    \PY{n}{qml}\PY{o}{.}\PY{n}{BasisState}\PY{p}{(}\PY{n}{u}\PY{p}{,} \PY{n}{wires}\PY{o}{=}\PY{p}{[}\PY{l+m+mi}{0}\PY{p}{]}\PY{p}{)}
    \PY{n}{qml}\PY{o}{.}\PY{n}{Hadamard}\PY{p}{(}\PY{n}{wires}\PY{o}{=}\PY{p}{[}\PY{l+m+mi}{0}\PY{p}{]}\PY{p}{)}
    \PY{n}{qml}\PY{o}{.}\PY{n}{PauliX}\PY{p}{(}\PY{n}{wires}\PY{o}{=}\PY{p}{[}\PY{l+m+mi}{0}\PY{p}{]}\PY{p}{)}
    \PY{k}{return} \PY{n}{qml}\PY{o}{.}\PY{n}{sample}\PY{p}{(}\PY{n}{qml}\PY{o}{.}\PY{n}{PauliX}\PY{p}{(}\PY{l+m+mi}{0}\PY{p}{)}\PY{p}{)}

\PY{n+nb}{print}\PY{p}{(}\PY{n}{circuit}\PY{p}{(}\PY{p}{)}\PY{p}{)}
\end{Verbatim}
\end{tcolorbox}

    \begin{Verbatim}[commandchars=\\\{\}]
-1
\end{Verbatim}

    Working through the linear algebra, we get the following:

\begin{align}
X\cdot H |1\rangle &= X \cdot H \cdot \begin{pmatrix} 0\\1 \end{pmatrix}\\
&= \frac{1}{\sqrt{2}}\begin{pmatrix}0&1\\1&0 \end{pmatrix}\begin{pmatrix}1&1\\1&-1 \end{pmatrix}\begin{pmatrix}0\\1 \end{pmatrix}\\
&= \frac{1}{\sqrt{2}}\begin{pmatrix}0&1\\1&0 \end{pmatrix}\begin{pmatrix}1\\-1 \end{pmatrix}\\
&= \frac{1}{\sqrt{2}}\begin{pmatrix}1\\-1 \end{pmatrix}\\
&= \frac{1}{\sqrt{2}}(|0\rangle -|1\rangle)\\
&= |l\rangle
\end{align}

So, we should indeed expect a sample value of ``\(-1\)'' when measuring
in the \(X\)-basis.

    Let's work through another example. Let's take the following diagram and
convert it to linear algebra:

\begin{quantikz}
\lstick{\ket{0} } & \gate{Y} & \gate{Z} & \qw |\psi \rangle
\end{quantikz}

    \begin{align}
    Z \cdot Y |0\rangle &= Z \cdot Y \cdot \begin{pmatrix}1\\0\end{pmatrix} \\
    &= \begin{pmatrix}1&0 \\ 0&-1 \end{pmatrix} \begin{pmatrix} 0 & -i \\ i & 0\end{pmatrix} \begin{pmatrix}1\\0\end{pmatrix}\\
    &= \begin{pmatrix}1&0 \\ 0&-1 \end{pmatrix}\begin{pmatrix}0\\i\end{pmatrix}\\
    &= \begin{pmatrix}0\\-i\end{pmatrix}\\
    &= |\psi \rangle
\end{align}

    Notice,

\begin{align}
\begin{pmatrix}
0\\-i
\end{pmatrix} = 
-i \begin{pmatrix}
0 \\ 1
\end{pmatrix} = 
-i |1\rangle 
\end{align}

We can write a circuit that performs these operations, and then samples
the circuit in the \(Z\)-basis:

    \begin{tcolorbox}[breakable, size=fbox, boxrule=1pt, pad at break*=1mm,colback=cellbackground, colframe=cellborder]
\prompt{In}{incolor}{4}{\hspace{4pt}}
\begin{Verbatim}[commandchars=\\\{\}]
\PY{c+c1}{\PYZsh{}Create the qnode}
\PY{n+nd}{@qml}\PY{o}{.}\PY{n}{qnode}\PY{p}{(}\PY{n}{dev}\PY{p}{)}
\PY{k}{def} \PY{n+nf}{circuit}\PY{p}{(}\PY{p}{)}\PY{p}{:}
    \PY{n}{qml}\PY{o}{.}\PY{n}{PauliY}\PY{p}{(}\PY{n}{wires}\PY{o}{=}\PY{p}{[}\PY{l+m+mi}{0}\PY{p}{]}\PY{p}{)}
    \PY{n}{qml}\PY{o}{.}\PY{n}{PauliZ}\PY{p}{(}\PY{n}{wires}\PY{o}{=}\PY{p}{[}\PY{l+m+mi}{0}\PY{p}{]}\PY{p}{)}
    \PY{k}{return} \PY{n}{qml}\PY{o}{.}\PY{n}{sample}\PY{p}{(}\PY{n}{qml}\PY{o}{.}\PY{n}{PauliZ}\PY{p}{(}\PY{l+m+mi}{0}\PY{p}{)}\PY{p}{)}

\PY{n+nb}{print}\PY{p}{(}\PY{n}{circuit}\PY{p}{(}\PY{p}{)}\PY{p}{)}
\end{Verbatim}
\end{tcolorbox}

    \begin{Verbatim}[commandchars=\\\{\}]
-1
\end{Verbatim}

    Due to the way measurements behave, we are always going to get a value
of ``\(-1\)'' when sampling the state \(-i|1\rangle\). We'll explain why
later on. We can write a circuit that performs these operations, and
then samples the circuit in the \(X\)-basis:

    \begin{tcolorbox}[breakable, size=fbox, boxrule=1pt, pad at break*=1mm,colback=cellbackground, colframe=cellborder]
\prompt{In}{incolor}{5}{\hspace{4pt}}
\begin{Verbatim}[commandchars=\\\{\}]
\PY{c+c1}{\PYZsh{}Create the qnode}
\PY{n+nd}{@qml}\PY{o}{.}\PY{n}{qnode}\PY{p}{(}\PY{n}{dev}\PY{p}{)}
\PY{k}{def} \PY{n+nf}{circuit}\PY{p}{(}\PY{p}{)}\PY{p}{:}
    \PY{n}{qml}\PY{o}{.}\PY{n}{PauliY}\PY{p}{(}\PY{n}{wires}\PY{o}{=}\PY{p}{[}\PY{l+m+mi}{0}\PY{p}{]}\PY{p}{)}
    \PY{n}{qml}\PY{o}{.}\PY{n}{PauliZ}\PY{p}{(}\PY{n}{wires}\PY{o}{=}\PY{p}{[}\PY{l+m+mi}{0}\PY{p}{]}\PY{p}{)}
    \PY{k}{return} \PY{n}{qml}\PY{o}{.}\PY{n}{sample}\PY{p}{(}\PY{n}{qml}\PY{o}{.}\PY{n}{PauliX}\PY{p}{(}\PY{l+m+mi}{0}\PY{p}{)}\PY{p}{)}

\PY{n+nb}{print}\PY{p}{(}\PY{n}{circuit}\PY{p}{(}\PY{p}{)}\PY{p}{)}
\end{Verbatim}
\end{tcolorbox}

    \begin{Verbatim}[commandchars=\\\{\}]
-1
\end{Verbatim}

    In this \texttt{qnode()}, the \(X\)-basis sample value is a quantum
phenomenon not seen in classical phsyics. In fact if we run the circuit
several times by defining a new ``20-shot device'' we will see that we
get a mixture of ``\(+1\)'' and ``\(-1\)'' sample values:

    \begin{tcolorbox}[breakable, size=fbox, boxrule=1pt, pad at break*=1mm,colback=cellbackground, colframe=cellborder]
\prompt{In}{incolor}{6}{\hspace{4pt}}
\begin{Verbatim}[commandchars=\\\{\}]
\PY{c+c1}{\PYZsh{} Create a device to run the code}
\PY{n}{dev2} \PY{o}{=} \PY{n}{qml}\PY{o}{.}\PY{n}{device}\PY{p}{(}\PY{l+s+s2}{\PYZdq{}}\PY{l+s+s2}{default.qubit}\PY{l+s+s2}{\PYZdq{}}\PY{p}{,} \PY{n}{wires}\PY{o}{=}\PY{l+m+mi}{1}\PY{p}{,} \PY{n}{shots}\PY{o}{=}\PY{l+m+mi}{20}\PY{p}{)}

\PY{c+c1}{\PYZsh{}Create the qnode}
\PY{n+nd}{@qml}\PY{o}{.}\PY{n}{qnode}\PY{p}{(}\PY{n}{dev2}\PY{p}{)}
\PY{k}{def} \PY{n+nf}{circuit}\PY{p}{(}\PY{p}{)}\PY{p}{:}
    \PY{n}{qml}\PY{o}{.}\PY{n}{PauliY}\PY{p}{(}\PY{n}{wires}\PY{o}{=}\PY{p}{[}\PY{l+m+mi}{0}\PY{p}{]}\PY{p}{)}
    \PY{n}{qml}\PY{o}{.}\PY{n}{PauliZ}\PY{p}{(}\PY{n}{wires}\PY{o}{=}\PY{p}{[}\PY{l+m+mi}{0}\PY{p}{]}\PY{p}{)}
    \PY{k}{return} \PY{n}{qml}\PY{o}{.}\PY{n}{sample}\PY{p}{(}\PY{n}{qml}\PY{o}{.}\PY{n}{PauliX}\PY{p}{(}\PY{l+m+mi}{0}\PY{p}{)}\PY{p}{)}

\PY{n+nb}{print}\PY{p}{(}\PY{n}{circuit}\PY{p}{(}\PY{p}{)}\PY{p}{)}
\end{Verbatim}
\end{tcolorbox}

    \begin{Verbatim}[commandchars=\\\{\}]
[ 1 -1 -1  1  1  1 -1  1 -1  1  1  1 -1 -1  1 -1  1  1  1  1]
\end{Verbatim}

    Explaining this strange behavior of measuring something different each
time and not getting a single sample value for every measurement is
something we will dig into when discussing measurements. Even though we
can work out a unique pure state using linear algebra, measurements
behave in a way me might not intially expect. In order to get the uniqe
pure state out of the measurement, we have to perform a measurement in
the basis defined by that state.

    Let's have a look at a more complicated circuit involving two qubits
now:

\begin{quantikz}
\lstick{\ket{0}} \qw & \ctrl{1} & \qw \rstick[wires=2]{$\ket{\psi}$} \\
\lstick{\ket{0}} \qw & \targ{} & \qw 
\end{quantikz}

    Let's work out the linear algebra:

\begin{align}
\mathbf{CX}|00\rangle 
&= 
\begin{pmatrix} 
1 & 0 & 0 & 0 \\
0 & 1 & 0 & 0 \\
0 & 0 & 0 & 1 \\
0 & 0 & 1 & 0
\end{pmatrix}
\begin{pmatrix} 1\\0\\0\\0 \end{pmatrix} \\
&= \begin{pmatrix} 1\\0\\0\\0 \end{pmatrix} \\
&= |00\rangle \\
&= |\psi \rangle 
\end{align}

    Now, let's write a circuit that performs these operations and samples
the circuit in the \(Z\)-basis one time. We will need a device with two
wires, and we have to tell the \texttt{qml.CNOT()} function which wires
to operate on:

    \begin{tcolorbox}[breakable, size=fbox, boxrule=1pt, pad at break*=1mm,colback=cellbackground, colframe=cellborder]
\prompt{In}{incolor}{7}{\hspace{4pt}}
\begin{Verbatim}[commandchars=\\\{\}]
\PY{c+c1}{\PYZsh{} Create a device to run the code}
\PY{n}{dev3} \PY{o}{=} \PY{n}{qml}\PY{o}{.}\PY{n}{device}\PY{p}{(}\PY{l+s+s2}{\PYZdq{}}\PY{l+s+s2}{default.qubit}\PY{l+s+s2}{\PYZdq{}}\PY{p}{,} \PY{n}{wires}\PY{o}{=}\PY{l+m+mi}{2}\PY{p}{,} \PY{n}{shots}\PY{o}{=}\PY{l+m+mi}{1}\PY{p}{)}

\PY{c+c1}{\PYZsh{}Create the qnode}
\PY{n+nd}{@qml}\PY{o}{.}\PY{n}{qnode}\PY{p}{(}\PY{n}{dev3}\PY{p}{)}
\PY{k}{def} \PY{n+nf}{circuit}\PY{p}{(}\PY{p}{)}\PY{p}{:}
    \PY{n}{qml}\PY{o}{.}\PY{n}{CNOT}\PY{p}{(}\PY{n}{wires}\PY{o}{=}\PY{p}{[}\PY{l+m+mi}{0}\PY{p}{,}\PY{l+m+mi}{1}\PY{p}{]}\PY{p}{)}
    \PY{k}{return} \PY{n}{qml}\PY{o}{.}\PY{n}{sample}\PY{p}{(}\PY{n}{qml}\PY{o}{.}\PY{n}{PauliZ}\PY{p}{(}\PY{l+m+mi}{0}\PY{p}{)}\PY{p}{)}

\PY{n+nb}{print}\PY{p}{(}\PY{n}{circuit}\PY{p}{(}\PY{p}{)}\PY{p}{)}
\end{Verbatim}
\end{tcolorbox}

    \begin{Verbatim}[commandchars=\\\{\}]
1
\end{Verbatim}

    Let's do the almost same thing but lets make the following change to the
circuit:

\begin{quantikz}
\lstick{\ket{1}} \qw & \ctrl{1} & \qw \rstick[wires=2]{$\ket{\psi}$} \\
\lstick{\ket{0}} \qw & \targ{} & \qw 
\end{quantikz}

    This is prepared in the \(|10 \rangle\) basis state, so we should expect
a different outcome state. Let's have a look at the math:

\begin{align}
\mathbf{CX}|10\rangle &= 
\begin{pmatrix} 
1 & 0 & 0 & 0 \\
0 & 1 & 0 & 0 \\
0 & 0 & 0 & 1 \\
0 & 0 & 1 & 0
\end{pmatrix}
\begin{pmatrix} 0\\0\\1\\0 \end{pmatrix}\\
&= \begin{pmatrix} 0\\0\\0\\1 \end{pmatrix}\\
&= |11\rangle \\
&= |\psi \rangle
\end{align}

As we can see, the second qubit value is flipped now. Let's build a
circuit that performs these operations an samples in the \(Z\)-basis on
both qubits once:

    \begin{tcolorbox}[breakable, size=fbox, boxrule=1pt, pad at break*=1mm,colback=cellbackground, colframe=cellborder]
\prompt{In}{incolor}{8}{\hspace{4pt}}
\begin{Verbatim}[commandchars=\\\{\}]
\PY{c+c1}{\PYZsh{} Create a device to run the code}
\PY{n}{dev3} \PY{o}{=} \PY{n}{qml}\PY{o}{.}\PY{n}{device}\PY{p}{(}\PY{l+s+s2}{\PYZdq{}}\PY{l+s+s2}{default.qubit}\PY{l+s+s2}{\PYZdq{}}\PY{p}{,} \PY{n}{wires}\PY{o}{=}\PY{l+m+mi}{2}\PY{p}{,} \PY{n}{shots}\PY{o}{=}\PY{l+m+mi}{1}\PY{p}{)}

\PY{c+c1}{\PYZsh{} Define an array corresponding to the initial basis state:}
\PY{n}{ud} \PY{o}{=} \PY{n}{np}\PY{o}{.}\PY{n}{array}\PY{p}{(}\PY{p}{[}\PY{l+m+mi}{1}\PY{p}{,}\PY{l+m+mi}{0}\PY{p}{]}\PY{p}{)}

\PY{c+c1}{\PYZsh{}Create the qnode}
\PY{n+nd}{@qml}\PY{o}{.}\PY{n}{qnode}\PY{p}{(}\PY{n}{dev3}\PY{p}{)}
\PY{k}{def} \PY{n+nf}{circuit}\PY{p}{(}\PY{p}{)}\PY{p}{:}
    \PY{n}{qml}\PY{o}{.}\PY{n}{BasisState}\PY{p}{(}\PY{n}{ud}\PY{p}{,} \PY{n}{wires}\PY{o}{=}\PY{p}{[}\PY{l+m+mi}{0}\PY{p}{,}\PY{l+m+mi}{1}\PY{p}{]}\PY{p}{)}
    \PY{n}{qml}\PY{o}{.}\PY{n}{CNOT}\PY{p}{(}\PY{n}{wires}\PY{o}{=}\PY{p}{[}\PY{l+m+mi}{0}\PY{p}{,}\PY{l+m+mi}{1}\PY{p}{]}\PY{p}{)}
    \PY{k}{return} \PY{n}{qml}\PY{o}{.}\PY{n}{sample}\PY{p}{(}\PY{n}{qml}\PY{o}{.}\PY{n}{PauliZ}\PY{p}{(}\PY{l+m+mi}{0}\PY{p}{)}\PY{p}{)}\PY{p}{,} \PY{n}{qml}\PY{o}{.}\PY{n}{sample}\PY{p}{(}\PY{n}{qml}\PY{o}{.}\PY{n}{PauliZ}\PY{p}{(}\PY{l+m+mi}{1}\PY{p}{)}\PY{p}{)}

\PY{n+nb}{print}\PY{p}{(}\PY{n}{circuit}\PY{p}{(}\PY{p}{)}\PY{p}{)}
\end{Verbatim}
\end{tcolorbox}

    \begin{Verbatim}[commandchars=\\\{\}]
[[-1]
 [-1]]
\end{Verbatim}

    We can see, since the final state is \(|\psi \rangle = |11\rangle\), we
get ``\(-1\)'' sample values for the functions

\begin{verbatim}
qml.sample(qml.PauliZ(0))
qml.sample(qml.PauliZ(1))
\end{verbatim}

    In general, if we perform the \textbf{CNOT} operation on two qubits in
some initial basis state we get the following output states:

\begin{align}
\mathbf{CX}|00 \rangle &= |00\rangle \\
\mathbf{CX}|01 \rangle &= |01\rangle \\
\mathbf{CX}|10 \rangle &= |11\rangle \\
\mathbf{CX}|11 \rangle &= |10\rangle
\end{align}

    It is important to mention at this point that the following circuit
diagram:

\begin{quantikz}
\lstick{\ket{0}} \qw & \gate{H} & \qw \rstick[wires=2]{$\ket{\psi}$} \\
\lstick{\ket{0}} \qw & \gate{I} & \qw 
\end{quantikz}

    Is equivalent to this diagram:

\begin{quantikz}
\lstick{\ket{0}} \qw & \gate{H} & \qw \rstick[wires=2]{$\ket{\psi}$} \\
\lstick{\ket{0}} \qw & \qw & \qw 
\end{quantikz}

    Much more complicated examples can of course be created, but the general
idea remains the same. Absence of a gate means the identity gate. We
generally can stack gates to get their tensor product. For example the
following circuit:

\begin{quantikz}
\lstick{\ket{0}} \qw & \gate{H} & \qw \rstick[wires=2]{$\ket{\psi}$} \\
\lstick{\ket{0}} \qw & \gate{H} & \qw 
\end{quantikz}

    Can be interpreted in terms of linear algebra as follows:

\begin{align}
(H \otimes H)|00\rangle &= 
\frac{1}{\sqrt{2}}
\begin{pmatrix}
1 & 1 \\
1 & -1
\end{pmatrix} \otimes 
\frac{1}{\sqrt{2}}
\begin{pmatrix}
1 & 1 \\
1 & -1
\end{pmatrix} 
\left(
\begin{pmatrix}
1 \\ 0
\end{pmatrix} \otimes 
\begin{pmatrix}
1 \\ 0
\end{pmatrix}
\right) \\
&= \frac{1}{2}
\begin{pmatrix}
1 & 1 & 1 & 1 \\
1 & -1 & 1 & -1 \\
1 & 1  & -1 & -1 \\
1 & -1 & -1 & 1
\end{pmatrix}
\begin{pmatrix}
1 \\ 0 \\ 0 \\ 0
\end{pmatrix} \\
&= \begin{pmatrix} 1/2 \\ 1/2 \\ 1/2 \\ 1/2 \end{pmatrix}
\end{align}

    This can also be computer in a slightly different but completely
equivalent way which may be easier for some people to parse:

    \begin{align}
(H \otimes H)|00\rangle &= H|0\rangle \otimes H|0 \rangle \\
&= \frac{1}{\sqrt{2}}
\begin{pmatrix}
1 & 1 \\
1 & -1
\end{pmatrix} \begin{pmatrix} 1\\0 \end{pmatrix} \otimes 
\frac{1}{\sqrt{2}}
\begin{pmatrix}
1 & 1 \\
1 & -1
\end{pmatrix} \begin{pmatrix} 1\\0 \end{pmatrix} \\
&= \frac{1}{2}\left(
\begin{pmatrix} 1\\1 \end{pmatrix} \otimes
\begin{pmatrix} 1\\1 \end{pmatrix}
\right) \\
&= \begin{pmatrix} 1/2 \\ 1/2 \\ 1/2 \\ 1/2 \end{pmatrix}
\end{align}

    Let's implement this in PennyLane but with the initial basis state
\(|10\rangle\):

    \begin{tcolorbox}[breakable, size=fbox, boxrule=1pt, pad at break*=1mm,colback=cellbackground, colframe=cellborder]
\prompt{In}{incolor}{9}{\hspace{4pt}}
\begin{Verbatim}[commandchars=\\\{\}]
\PY{c+c1}{\PYZsh{} Create a device to run the code}
\PY{n}{dev3} \PY{o}{=} \PY{n}{qml}\PY{o}{.}\PY{n}{device}\PY{p}{(}\PY{l+s+s2}{\PYZdq{}}\PY{l+s+s2}{default.qubit}\PY{l+s+s2}{\PYZdq{}}\PY{p}{,} \PY{n}{wires}\PY{o}{=}\PY{l+m+mi}{2}\PY{p}{,} \PY{n}{shots}\PY{o}{=}\PY{l+m+mi}{1}\PY{p}{)}

\PY{c+c1}{\PYZsh{} Define an array corresponding to the initial basis state:}
\PY{n}{ud} \PY{o}{=} \PY{n}{np}\PY{o}{.}\PY{n}{array}\PY{p}{(}\PY{p}{[}\PY{l+m+mi}{1}\PY{p}{,}\PY{l+m+mi}{0}\PY{p}{]}\PY{p}{)}

\PY{c+c1}{\PYZsh{}Create the qnode}
\PY{n+nd}{@qml}\PY{o}{.}\PY{n}{qnode}\PY{p}{(}\PY{n}{dev3}\PY{p}{)}
\PY{k}{def} \PY{n+nf}{circuit}\PY{p}{(}\PY{p}{)}\PY{p}{:}
    \PY{n}{qml}\PY{o}{.}\PY{n}{BasisState}\PY{p}{(}\PY{n}{ud}\PY{p}{,} \PY{n}{wires}\PY{o}{=}\PY{p}{[}\PY{l+m+mi}{0}\PY{p}{,}\PY{l+m+mi}{1}\PY{p}{]}\PY{p}{)}
    \PY{n}{qml}\PY{o}{.}\PY{n}{Hadamard}\PY{p}{(}\PY{n}{wires}\PY{o}{=}\PY{p}{[}\PY{l+m+mi}{0}\PY{p}{]}\PY{p}{)}
    \PY{n}{qml}\PY{o}{.}\PY{n}{Hadamard}\PY{p}{(}\PY{n}{wires}\PY{o}{=}\PY{p}{[}\PY{l+m+mi}{1}\PY{p}{]}\PY{p}{)}
    \PY{k}{return} \PY{n}{qml}\PY{o}{.}\PY{n}{sample}\PY{p}{(}\PY{n}{qml}\PY{o}{.}\PY{n}{PauliZ}\PY{p}{(}\PY{l+m+mi}{0}\PY{p}{)}\PY{p}{)}\PY{p}{,} \PY{n}{qml}\PY{o}{.}\PY{n}{sample}\PY{p}{(}\PY{n}{qml}\PY{o}{.}\PY{n}{PauliZ}\PY{p}{(}\PY{l+m+mi}{1}\PY{p}{)}\PY{p}{)}

\PY{n+nb}{print}\PY{p}{(}\PY{n}{circuit}\PY{p}{(}\PY{p}{)}\PY{p}{)}
\end{Verbatim}
\end{tcolorbox}

    \begin{Verbatim}[commandchars=\\\{\}]
[[-1]
 [-1]]
\end{Verbatim}

    Let's look at one final example that is more complicated:

\begin{quantikz}
\lstick{$\ket{0}$} & \gate{H} & \ctrl{1} & \gate{Y} \qw \rstick[wires=2]{$\ket{\psi}$} \\
\lstick{$\ket{0}$} & \gate{H} & \targ{}  & \gate{Z} \qw \\
\end{quantikz}

    This one is a little tricky if you don't know about tensor products so
let's go through it very carefully. First, operate on each qubit with
the Hadamard gate to turn the ``ket-\(0\)'' into a superposition:

\begin{align}
H |0\rangle = \frac{1}{\sqrt{2}}\begin{pmatrix} 1&1 \\ 1&-1\end{pmatrix}\begin{pmatrix}1\\0\end{pmatrix}  = \frac{1}{\sqrt{2}} \begin{pmatrix} 1\\1 \end{pmatrix}.
\end{align}

    Now, take the two qubits in superposition (two copies of the above
computation), and take their tensor product:

\begin{align}
    H|0\rangle \otimes H|0\rangle &= \frac{1}{\sqrt{2}}\begin{pmatrix} 1\\1 \end{pmatrix} \otimes \frac{1}{\sqrt{2}} \begin{pmatrix} 1\\1 \end{pmatrix}\\ 
    &= \frac{1}{2}\begin{pmatrix} 1\\1\\1\\1 \end{pmatrix}\\
    &= \begin{pmatrix} 1/2\\1/2\\1/2\\1/2 \end{pmatrix}
\end{align}

    Next, operate on this by the \(\mathbf{CNOT}\)-gate

\begin{align}
\begin{pmatrix} 1 & 0 & 0 & 0\\ 0 & 1 & 0 & 0 \\ 0 & 0 & 0 & 1\\ 0 & 0 & 1 & 0 \end{pmatrix} \begin{pmatrix} 1/2\\1/2\\1/2\\1/2 \end{pmatrix} = 
\begin{pmatrix} 1/2\\1/2\\1/2\\1/2 \end{pmatrix}
\end{align}

    Next, we operate on the tensor product vector with the operator
\(Y \otimes Z\):

\begin{align}
Y \otimes Z \begin{pmatrix}
1/2 \\ 1/2 \\ 1/2 \\ 1/2
\end{pmatrix} 
&= \begin{pmatrix}0&-i \\ i&0\end{pmatrix} \otimes \begin{pmatrix}1&0 \\ 0&-1 \end{pmatrix} \begin{pmatrix}
1/2 \\ 1/2 \\ 1/2 \\ 1/2
\end{pmatrix} \\
&= \begin{pmatrix}
0 & 0 & -i & 0 \\
0 & 0 & 0 & i \\
i & 0 & 0 & 0 \\
0 & -i & 0 & 0
\end{pmatrix}\begin{pmatrix}
1/2 \\ 1/2 \\ 1/2 \\ 1/2
\end{pmatrix} \\
&= \begin{pmatrix}
-i/2 \\ i/2 \\ i/2 \\ -i/2
\end{pmatrix}
\end{align}

    Now, let's implement some PennyLane code to do the same operations on
two qubits:

    \begin{tcolorbox}[breakable, size=fbox, boxrule=1pt, pad at break*=1mm,colback=cellbackground, colframe=cellborder]
\prompt{In}{incolor}{10}{\hspace{4pt}}
\begin{Verbatim}[commandchars=\\\{\}]
\PY{c+c1}{\PYZsh{} Create a device to run the code}
\PY{n}{dev3} \PY{o}{=} \PY{n}{qml}\PY{o}{.}\PY{n}{device}\PY{p}{(}\PY{l+s+s2}{\PYZdq{}}\PY{l+s+s2}{default.qubit}\PY{l+s+s2}{\PYZdq{}}\PY{p}{,} \PY{n}{wires}\PY{o}{=}\PY{l+m+mi}{2}\PY{p}{,} \PY{n}{shots}\PY{o}{=}\PY{l+m+mi}{1}\PY{p}{)}

\PY{c+c1}{\PYZsh{}Create the qnode}
\PY{n+nd}{@qml}\PY{o}{.}\PY{n}{qnode}\PY{p}{(}\PY{n}{dev3}\PY{p}{)}
\PY{k}{def} \PY{n+nf}{circuit}\PY{p}{(}\PY{p}{)}\PY{p}{:}
    \PY{n}{qml}\PY{o}{.}\PY{n}{Hadamard}\PY{p}{(}\PY{n}{wires}\PY{o}{=}\PY{p}{[}\PY{l+m+mi}{0}\PY{p}{]}\PY{p}{)}
    \PY{n}{qml}\PY{o}{.}\PY{n}{Hadamard}\PY{p}{(}\PY{n}{wires}\PY{o}{=}\PY{p}{[}\PY{l+m+mi}{1}\PY{p}{]}\PY{p}{)}
    \PY{n}{qml}\PY{o}{.}\PY{n}{CNOT}\PY{p}{(}\PY{n}{wires}\PY{o}{=}\PY{p}{[}\PY{l+m+mi}{0}\PY{p}{,}\PY{l+m+mi}{1}\PY{p}{]}\PY{p}{)}
    \PY{n}{qml}\PY{o}{.}\PY{n}{PauliY}\PY{p}{(}\PY{n}{wires}\PY{o}{=}\PY{p}{[}\PY{l+m+mi}{0}\PY{p}{]}\PY{p}{)}
    \PY{n}{qml}\PY{o}{.}\PY{n}{PauliZ}\PY{p}{(}\PY{n}{wires}\PY{o}{=}\PY{p}{[}\PY{l+m+mi}{1}\PY{p}{]}\PY{p}{)}
    \PY{k}{return} \PY{n}{qml}\PY{o}{.}\PY{n}{sample}\PY{p}{(}\PY{n}{qml}\PY{o}{.}\PY{n}{PauliZ}\PY{p}{(}\PY{l+m+mi}{0}\PY{p}{)}\PY{p}{)}\PY{p}{,} \PY{n}{qml}\PY{o}{.}\PY{n}{sample}\PY{p}{(}\PY{n}{qml}\PY{o}{.}\PY{n}{PauliZ}\PY{p}{(}\PY{l+m+mi}{1}\PY{p}{)}\PY{p}{)}

\PY{n+nb}{print}\PY{p}{(}\PY{n}{circuit}\PY{p}{(}\PY{p}{)}\PY{p}{)}
\end{Verbatim}
\end{tcolorbox}

    \begin{Verbatim}[commandchars=\\\{\}]
[[-1]
 [ 1]]
\end{Verbatim}

    \hypertarget{exercises}{%
\subsubsection{Exercises}\label{exercises}}

    \begin{enumerate}
\def\labelenumi{\arabic{enumi}.}
\tightlist
\item
  Write PennyLane code to construct and measure the following circuit in
  the \(Z\)-basis. Compute the linear algebra by hand to see the output
  states.
\end{enumerate}

\begin{quantikz}
\lstick{\ket{0}} \qw & \gate{H} & \gate{Z} & \qw \rstick[wires=1]{$\ket{\psi}$} 
\end{quantikz}

    \begin{enumerate}
\def\labelenumi{\arabic{enumi}.}
\setcounter{enumi}{1}
\tightlist
\item
  Write PennyLane code to construct and measure the following circuit in
  the \(Z\)-basis. Compute the linear algebra by hand to see the output
  states.
\end{enumerate}

\begin{quantikz}
\lstick{\ket{0}} \qw & \gate{X} & \gate{Y} & \qw \rstick[wires=1]{$\ket{\psi}$} 
\end{quantikz}

    \begin{enumerate}
\def\labelenumi{\arabic{enumi}.}
\setcounter{enumi}{2}
\tightlist
\item
  Write PennyLane code to construct and measure the following circuit in
  the \(Z\)-basis. Compute the linear algebra by hand to see the output
  states.
\end{enumerate}

\begin{quantikz}
\lstick{\ket{0}} \qw & \gate{X} & \gate{H} & \gate{Y} & \qw \rstick[wires=1]{$\ket{\psi}$} 
\end{quantikz}

    \begin{enumerate}
\def\labelenumi{\arabic{enumi}.}
\setcounter{enumi}{3}
\tightlist
\item
  Write PennyLane code to construct and measure the following circuit in
  the \(Z\)-basis. Compute the linear algebra by hand to see the output
  states.
\end{enumerate}

\begin{quantikz}
\lstick{\ket{0}} \qw & \gate{H} & \ctrl{1} & \qw \rstick[wires=2]{$\ket{\psi}$} \\
\lstick{\ket{0}} \qw & \gate{H} & \targ{} & \qw
\end{quantikz}

    \begin{enumerate}
\def\labelenumi{\arabic{enumi}.}
\setcounter{enumi}{4}
\tightlist
\item
  Write PennyLane code to construct and measure the following circuit in
  the \(Z\)-basis. Compute the linear algebra by hand to see the output
  states.
\end{enumerate}

\begin{quantikz}
\lstick{\ket{1}} \qw & \gate{X} & \gate{Z} & \qw \rstick[wires=2]{$\ket{\psi}$} \\
\lstick{\ket{0}} \qw & \gate{H} & \gate{Y} & \qw
\end{quantikz}

    \begin{enumerate}
\def\labelenumi{\arabic{enumi}.}
\setcounter{enumi}{5}
\tightlist
\item
  Write PennyLane code to construct and measure the following circuit in
  the \(Z\)-basis. Compute the linear algebra by hand to see the output
  states.
\end{enumerate}

\begin{quantikz}
\lstick{\ket{1}} \qw & \gate{X} & \gate{Z} & \qw \rstick[wires=2]{$\ket{\psi}$} \\
\lstick{\ket{1}} \qw & \gate{H} & \qw & \qw
\end{quantikz}





    % Add a bibliography block to the postdoc
    
    
    
    \end{document}
